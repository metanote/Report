\chapter{INTRODUCTION}
\paragraph{}
Au cours de ma première année de master Web Intelligence à l'université de Jean Monnet Saint-Etienne, nous devions effectuer un stage d'une durée minimale de trois mois à compter du 18 mars.
\subparagraph{}
Ce rapport présente le travail que j'ai effectué lors de mon stage au sein de Stample.
\newline
Ce stage a été une bonne opportunité, je me suis familiarisé avec l'environnement technique, un ensemble d'application Scala/PlayFrameWork et GitHub.
J'ai découvert le co-working et j'ai beaucoup appris sur la philosophie du web.
\subparagraph{}
Stample s'est avéré un projet intéressant et très enrichissant pour une expérience professionnelle. Gr\^ace à ce stage, j'ai travaillé sur diffèrentes fonctionnalitées d'un réseau social.
\subparagraph{}
Le but de ce rapport n’est pas de faire une présentation exhaustive de tous les aspects techniques que j’ai pu aborder de façon succinte ou approfondie, mais plutôt de faire un tour d’horizon des éléments techniques et humains les plus importants.
\subparagraph{}
Il apparait cohérent de commencer mon rapport de stage par une présentation de Stample et de son contexte technique. Ensuite, je décrirais l'intégration du module d'authentification "secureSocial", l'amélioration de l'interface admin  et le système de notifications. Enfin, je conclurais  mon travail sur les perspectives de Stample.