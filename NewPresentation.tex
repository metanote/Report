\chapter{Présentation de la Plateforme Stample}
\section{Présentation de Stample}
Stample, plateforme de réseau social, redonne à chacun le contrôle exclusif de ses informations et améliore l'ergonomie de l'apprentissage, du travail individuel et de la collaboration.
\subsection{Stample en quelques mots}
\begin{itemize}
\item \textbf{Sample (échantillon),} utiliser le script bookmarklet pour attraper facilement le contenu multimédia du web. Permettre le Drag \& drop des fichiers depuis votre ordinateur, créer des notes et articles de n'importe quel appareil.
\item \textbf{Staple (article de base),} gracieusement organiser votre bibliothèque de contenu personnel grâce à un desgin d'arborescence similaire à un système de fichier.
Options visuelles conçues avec soin des informations dont vous avez besoin et accessibles en un coup d'oeil.
\item \textbf{Stamp (cachet),} résumer, mettre en évidence et annoter tous vos contenus.
Partager parfaitement du contenu et des métadonnées (des catégories) avec certains membres de votre réseau.
\end{itemize}
\subsection{Effectif}
L'equipe Stample est conçu par les Co-founders :
\textit{Edward Silhol, Henry Story et Sacha Roger} une courte pésentation disponible sur le site de la plateforme\footnote{https://stample.co/}.\newline
\textit{Amélie Medem} chercheur, Phd en computer science. Elle travaille en temps complet sur le FrontEnd de Stample.\newline
\textit{Sébastien Lober} développeur ingénieur Backend, il travaille actuellement chez Zenika\footnote{http://www.zenika.com/} et en temps partiel pour Stample.\newline
\textit{Jonathan Winandy}, développeur ingénieur Backend, il travaille chez Viadeo\footnote{http://us.viadeo.com/} et en temps partiel sur Stample Plateforme.\newline
\textit{Matthieu gayon}, développeur Frontend il travaille en temps partiel sur la Plateforme.\newline
\textit{Moncef Ben Rajeb}, développeur stagiaire Backend.\newline
Ce sont les développeurs actuels de Stample mais il y avait d'autres telque \textit{Francesco Piccoli}, développeur Backend qui a travaillé sur ce projet, ect...
\subsection{Levée de fonds}
\paragraph{}
D'après Steve Blank\footnote{http://steveblank.com/} « Il y'a une différence fondamentale entre une entreprise établie et 
une Startup : une Startup cherche un business model alors qu’une entreprise, elle, a déjà un 
business model »\newline
Suite à des réunions avec des inverstisseurs "friends and family", les trois fondateurs ont réussi à levéer 200 mille euro.
Leur objectif maintenant est d'avoir une première démo pour continuer qui sera probablement au mois de septembre.
\subsection{Besoins}
Il y a un besoin croissant d'un outil de partage sécurisé des connaissances numérique. Les gens perdent des heures chaque semaine en raison de la complexité croissante de leur vie numérique:
\begin{itemize}
\item Filtrage des e-mails et notifications non désirés,
\item Créations et mises à jour de leurs profils sur de trop nombreux services isolés les uns des autres,
\item Récupération de mots de passe perdus ou volés, ect...
\item L'information pertinente est de plus en plus difficile à extraire du déluge de données,
\item La grande segmentation des outils rend l'organisation et la collaboration frustrante.
\item Améliorer l'ergonomie de l'apprentissage et du travail par la contextualisation avancée de l'information,

\end{itemize}
\subsection{Concurrence}
\begin{itemize}
\item Stockage en ligne: Dropbox, Box, Google Drive, etc…
\item Réseaux sociaux: Facebook, Tumblr, Pinterest, Instagram, etc…
\item Aggrégateurs et plateformes de blogs: WordPress, Twitter, Reddit, Scoop.it, Paper.li, Flipboard, Zite, Jolicloud, etc…
\item Réseaux sociaux professionnels: LinkedIn, Quora, Yammer, Podio, etc…
\item Outils de note et d'organisation d'informations: Evernote, SpringPad, Clipboard, Pearltrees, Kippt, Google Keep, etc…
\item Outils de partage des connaissances: Mendeley, Kno, Scribd, SlideShare, Issuu

\end{itemize}
\section{Conclusion du chapitre 1}

\paragraph{}
Stample est conçu pour s'adapter aux gestes et aux usages quotidiens des gens. Les utilisateurs de la plateforme pourront chacun construire leur réseau de connaissances personnelles, en évaluant leurs sources et les contenus qu'ils partagent.
Stample utilise la mécanique du jeu pour encourager les gens à partager des informations hautement qualifiées et contextualisés.
